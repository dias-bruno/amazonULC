\documentclass[preprint, 3p,
authoryear]{elsarticle} %review=doublespace preprint=single 5p=2 column
%%% Begin My package additions %%%%%%%%%%%%%%%%%%%

\usepackage[hyphens]{url}

  \journal{An awesome journal} % Sets Journal name

\usepackage{lineno} % add

\usepackage{graphicx}
%%%%%%%%%%%%%%%% end my additions to header

\usepackage[T1]{fontenc}
\usepackage{lmodern}
\usepackage{amssymb,amsmath}
\usepackage{ifxetex,ifluatex}
\usepackage{fixltx2e} % provides \textsubscript
% use upquote if available, for straight quotes in verbatim environments
\IfFileExists{upquote.sty}{\usepackage{upquote}}{}
\ifnum 0\ifxetex 1\fi\ifluatex 1\fi=0 % if pdftex
  \usepackage[utf8]{inputenc}
\else % if luatex or xelatex
  \usepackage{fontspec}
  \ifxetex
    \usepackage{xltxtra,xunicode}
  \fi
  \defaultfontfeatures{Mapping=tex-text,Scale=MatchLowercase}
  \newcommand{\euro}{€}
\fi
% use microtype if available
\IfFileExists{microtype.sty}{\usepackage{microtype}}{}
\usepackage[]{natbib}
\bibliographystyle{plainnat}

\ifxetex
  \usepackage[setpagesize=false, % page size defined by xetex
              unicode=false, % unicode breaks when used with xetex
              xetex]{hyperref}
\else
  \usepackage[unicode=true]{hyperref}
\fi
\hypersetup{breaklinks=true,
            bookmarks=true,
            pdfauthor={},
            pdftitle={Short Paper},
            colorlinks=false,
            urlcolor=blue,
            linkcolor=magenta,
            pdfborder={0 0 0}}

\setcounter{secnumdepth}{5}
% Pandoc toggle for numbering sections (defaults to be off)


% tightlist command for lists without linebreak
\providecommand{\tightlist}{%
  \setlength{\itemsep}{0pt}\setlength{\parskip}{0pt}}






\begin{document}


\begin{frontmatter}

  \title{Short Paper}
    \author[Some Institute of Technology]{Alice Anonymous%
  \corref{cor1}%
  \fnref{1}}
   \ead{alice@example.com} 
    \author[Another University]{Bob Security%
  %
  }
   \ead{bob@example.com} 
    \author[Another University]{Cat Memes%
  %
  \fnref{2}}
   \ead{cat@example.com} 
    \author[Some Institute of Technology]{Derek Zoolander%
  %
  \fnref{2}}
   \ead{derek@example.com} 
      \affiliation[Some Institute of Technology]{Department, Street,
City, State, Zip}
    \affiliation[Another University]{Department, Street, City, State,
Zip}
    \cortext[cor1]{Corresponding author}
    \fntext[1]{This is the first author footnote.}
    \fntext[2]{Another author footnote.}
  
  \begin{abstract}
  The primary method for collecting information about the Earth's
  surface in recent decades, notably for developing nations, has been
  remote sensing. Despite this, Amazonian cities lack databases and
  cartographic publications. Considering Santarém as the study site,
  this paper proposes to create a classification model for mapping the
  land cover of an Amazonian city. Using imagery from the CBERS-4A
  satellite's WPM sensor, we created a classification model that
  combines the Geographic Object-Based Image Analysis (GEOBIA) method,
  data mining strategies, and the Random Forest machine learning
  algorithm. The results are promising in discerning different
  intra-urban cover classes, with an overall accuracy level in the
  validation samples of over 98\%.
  \end{abstract}
    \begin{keyword}
    keyword1 \sep 
    keyword2
  \end{keyword}
  
 \end{frontmatter}

\hypertarget{bibliography-styles}{%
\section{Bibliography styles}\label{bibliography-styles}}

In recent decades, remote sensing has been the main basis for collecting
data about the Earth's surface, especially for developing countries
\citep{zhu2022urban}. The data obtained through remote sensing offers
numerous opportunities for mapping and monitoring cities. They serve as
the basis for physical, climatic, and socio-economic indicators, provide
consistent quantitative data over time and space, and are an alternative
to traditional surveys - such as those of the census. Once this data is
translated into information and regularly updated, it can support urban
planning {[}2{]}.

\renewcommand\refname{References}
\bibliography{references.bib}


\end{document}
